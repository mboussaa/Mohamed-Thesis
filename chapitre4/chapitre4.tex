
The intensive use of generative programming techniques provides an elegant engineering solution to deal with the heterogeneity of platforms and technological stacks. The use of domain-specific languages for example, leads to the creation of numerous code generators that automatically translate high-level system specifications into multi-target executable code. 

Producing correct and efficient code generator is complex and error-prone. Although software designers provide generally high-level test suites to verify the functional outcome of generated code, it remains challenging and tedious to verify the behavior of produced code in terms of non-functional properties.

This chapter describes a black-box testing approach that automatically detect anomalies in code generators in terms of non-functional properties (i.e., resource usage and performance).
 
In fact, we adapt the idea of metamorphic testing to the problem of code generators testing.  Hence, our approach relies on the definition of high-level test oracles (i.e., metamorphic relations) to check the potential inefficient code generator among a family of code generators. 

We evaluate our approach by analyzing the performance of Haxe, a popular high-level programming language that involves a set of cross-platform code generators. 
%Experimental results show that our approach is able to detect some performance inconsistencies that reveal real issues in Haxe code generators.

This chapter is organized as follows: 

Section \ref{sec:cg_introduction} introduces the context of this work, i.e., the non-functional testing of code generators.

Section \ref{sec:cg_Motivation} presents the motivation and background of this work. In particular, we discuss in this section three motivation examples and the problems we are addressing.

Section \ref{sec:cd_approach} describes the general approach overview and the testing strategy.

In Section \ref{sec:cg_evaluation}, the evaluation and results of our experiments are discussed. Hence, we provide more details about the experimental settings, the code generators under test, the benchmark used, the evaluation metrics, etc. We discuss then the evaluation results.

Finally, we conclude in Section \ref{sec:cg-conclusion}. 


\section{Introduction}
\label{sec:cg_introduction}

Generative programming techniques become a common practice for software development to tame the runtime platform heterogeneity that exists in several domains such as mobile or Internet of Things development. 

The main benefit of using generative programming is to reduce the development and maintenance effort, allowing the development at a higher-level of abstraction through the use of Domain-Specific Languages (DSLs)~\cite{brambilla2012model} for example. 

DSLs, as opposed to general-purpose languages, are high level software languages that focus on specific problem domains. 
DSLs or models are generally coupled with the use of code generators that will automatically transform the manually designed models to software artifacts, which can be deployed on different target platforms. 

However, code generators are known to be very difficult to implement and maintain since they involve a set of complex and heterogeneous technologies~\cite{france2007model,guana2015developers}.

To preserve software reliability and quality, code generators have to respect different requirements. In fact, \textit{non-mature} code generators can generate defective software artifacts which range from uncompilable or semantically dysfunctional code that causes serious damage to the generated software; to non-functional bugs which lead to poor-quality code that can affect system reliability and performance (\eg, high resource usage, high execution time, etc.). 

As a matter of fact, these defects (or also anomalies) should be detected and corrected as early as possible in order to ensure the correct behavior of delivered software.

To check the correctness of the code generation process, developers often define (at design or runtime level) a set of test cases that will verify the functional behavior of generated code. 

After code generation, test suites are executed within each target platform, which may lead to either a correct behavior (\ie, expected output) or incorrect one (\ie, failures, errors).

However, the functional correctness of generated code is not enough to claim the effectiveness of code generators. Properties such as memory usage and performance are very important to evaluate. In some cases, the quality of the generated code can negatively influence on the non-functional requirements and cause performance issues\cite{hundt2011loop,ray2014large}.

Testing the non-functional properties of code generators is a challenging and time-consuming task because developers need to deploy and run code every time a change is made in order to analyze and verify its non-functional behavior.

This task becomes more tedious when targeting different platforms and software languages. Thus, different platform-specific tools will be needed to track bugs and identify the cause of execution failures~\cite{guana2014chaintracker,delgado2004taxonomy}. 

As stated in the state of the art, there is a lack of automatic solutions that check the non-functional issues such as the properties related to the resource consumption of generated code (Memory or CPU consumption).

%we propose a testing approach based on a runtime monitoring infrastructure to automatically check the potential inefficient code generators. This infrastructure, based on system containers as execution platforms, allows code-generator developers to evaluate the generated code performance. %In fact, we provide a fine-grained understanding of resource consumption and analysis of components behavior. 


In this chapter, we are presenting the following contributions:

\begin{itemize} 	
	
	\item We propose a fully automated black box testing approach for detecting code generator inconsistencies within code generator families. We use metamorphic relations as means of test oracles for our test suites. In this contribution, we focus on detecting anomalies related to performance and resource usage properties.
	
	\item We report the results of an empirical study by evaluating the non-functional properties of the Haxe code generators. 
	Haxe is a popular high-level programming language\footnote{\num{1442} GitHub stars} that involves a set of cross-platform code generators able to generate code to different target platforms. The obtained results provide evidence to support the claim that our proposed approach is able to detect code generator issues.
	

	
\end{itemize}

\section{Context and motivations}
\label{sec:cg_Motivation}

\subsection{Code generator families}
When confronted with the requirement to generate code for a wide range of languages, middleware, libraries, hardware architectures, and operating systems, different customizable code generators can be used to easily and efficiently  generate code for different platforms.

This work is based on the intuition that a code generator is often a member of a family of code generators\cite{chae2008building}.

\begin{mydef}[\textbf{Code generator family}]
	We define a code generator family as a set of code generators that takes as input the same language/model and generate code for different target platforms.
\end{mydef}
%Thus, we can automatically compare the performance between different versions
%of generated code (coming from the same source program). Based on this comparison, we can automatically detect singular resource consumptions that could reveal a code generator bug.

As motivating examples for this research work, we can cite three approaches that intensively develop and use code generator families: 
\paragraph{a. Haxe.} 	Haxe\footnote{\url{http://haxe.org/}}~\cite{dasnois2011haxe} is an open source toolkit for cross-platform development which compiles to a number of different programming platforms, including JavaScript, Flash, PHP, C++, C\#, and Java. Haxe involves many features: the Haxe language, multi-platform compilers, and different native libraries. The Haxe language is a high-level programming language which is strictly typed. This language supports both, functional and object-oriented programming paradigms. It has a common type hierarchy, making certain API available on every target platform. Moreover, Haxe comes with a set of code generators that translate manually-written code (in Haxe language) to different target languages and platforms.  
%Haxe code can be compiled for applications running on desktop, mobile, and web platforms. Compilers ensure the correctness of user code in terms of syntax and type safety. Haxe comes also with a set of standard libraries that can be used on all supported targets and platform-specific libraries for each of them. One of the main uses of Haxe is to develop cross-platform games or cross-platform libraries that can run on mobile, on the Web or on a Desktop.  
This project is popular (more than \num{1440} stars on GitHub).

\paragraph{b. ThingML.} ThingML\footnote{\url{http://thingml.org/}} is a modeling language for embedded and distributed systems~\cite{fleurey2011mde}. The idea of ThingML is to develop a practical model-driven software engineering tool-chain which targets resource-constrained embedded systems such as low-power sensors and microcontroller-based devices. ThingML is developed as a domain-specific modeling language which includes concepts to describe both software components and communication protocols. The formalism used is a combination of architecture models, state machines and an imperative action language. The ThingML tool-set provides a  code generator families  to translate ThingML to C, Java and JavaScript. It includes a set of variants for the C and JavaScript code generators to target different embedded systems and their constraints. 
This project is still confidential, but it is a good candidate to represent the modeling community practices.

\paragraph{c. TypeScript.} TypeScript\footnote{\url{https://www.typescriptlang.org/}}is a typed superset of JavaScript that compiles to plain JavaScript~\cite{rastogi2015safe}. In fact, it does not compile to only one version of JavaScript. It can transform TypeScript to EcmaScript 3, 5 or 6. It can generate JavaScript that uses different system modules ('none', 'commonjs', 'amd', 'system', 'umd', 'es6', or 'es2015')\footnote{Each of this variation point can target different code generators (function \textit{emitES6Module} vs \textit{emitUMDModule} in emitter.ts for example).}. 
This project is popular (more than \num{12619} stars on GitHub).

Functionally testing a code generator family in this case would be simple. Since the generated programs have the same input program, the oracle can be defined as the comparison between the functional outputs of these programs which should be the same.
In fact, based on the three sample projects presented above, we remark that all GitHub code repositories of the corresponding projects use unit tests to check the correctness of code generators.  

In terms of non-functional tests, we observe that ThingML and TypeScript do not provide any specific tests to check the consistency of code generators regarding the memory or CPU usage properties. Haxe provides two test cases\footnote{https://github.com/HaxeFoundation/haxe/tree/development/tests/benchs} to benchmark the resulting generated code. One serves to benchmark an example in which object allocations are deliberately (over) used to measure how memory access/GC mixes with numeric processing in different target languages. The second test evaluates the network speed across different target platforms.


\subsection{Issues when testing a code generator family}

The main difficulties with testing the resource usage properties of code generators is that we cannot just observe the execution of produced code, but we have to observe and compare the execution of generated programs with equivalent (or reference) implementations (i.e., in other languages). Even if there is no explicit oracle to detect inconsistencies for a single code generator, we could benefit from the family of code generators to compare the behavior of several generated programs and detect singular resource consumption profiles that could reveal a code generator inconsistency~\cite{hundt2011loop}. 

As a consequence, we define a code generator inconsistency as:

\begin{mydef}[\textbf{code generator inconsistency}]
	
	A code generator that produces code which has a singular behavior in terms of performance or resource usage compared to all equivalent implementations in the same family.
\end{mydef}

The potential issues that can result in code generator inconsistencies can be resumed as following:
\begin{itemize}
	\setlength\itemsep{0em}
	\item  the lack of use of a \textbf{specific function that exists in the standard library} of the target API language  that can speed or reduce the memory consumption of the resulting program.
	\item the lack of use of \textbf{a specific type that exists in the standard library} of the target language  that can speed or reduce the memory consumption of the resulting program.
	\item  the lack of use of\textbf{ a specific language feature in a target language} that can speed or reduce the memory consumption of the resulting program. 
\end{itemize}

Next section discusses the common process used by developers to automatically test the performance of generated code. We also illustrate how we can benefit from the code generators families to identify suspect singular behaviors.


%In short, then, we believe that testing the non-functional properties of code generators remains challenging and time-consuming task because developers have to analyze and verify code for each target platform using platform-dependent tools which makes the task of maintaining code generators very tedious. The heterogeneity of platforms and the diversity of target software languages increase the need of supporting tools that can evaluate the consistency and coherence of generated code regarding the non-functional properties. This paper describes a new approach, based on micro-services as execution platforms, to automate and ease the non-functional testing of code generators. This runtime monitoring infrastructure provides a fine-grained understanding of resource consumption and analysis of generated code's behavior.

\section{The traditional process for non-functional testing of a code generator family}
A reliable and acceptable way to increase the confidence in the correctness of a code generator family is to validate and check the functionality of generated code, which is a common practice for compiler validation and testing~\cite{jorges2014back,stuermer2007systematic,sturmer2005overview}.
%Therefore, developers try to check the syntactic and semantic correctness of the generated code by means of different techniques such as static analysis, test suites, etc., and ensure that the code is behaving correctly.  
However, proving that the generated code is functionally correct is not enough to claim the effectiveness of the code generator under test. 

In fact, code generators have to respect different requirements to preserve software reliability and quality~\cite{demertzi2011analyzing}. 
In this case, ensuring the code quality of generated code requires examining several non-functional properties such as code size, resource or energy consuption, execution time, etc~\cite{pan2006fast}.
A \textit{non-mature} code generator might generate defective software artifacts (code smells) that violates common software engineering practices, resulting in a poor-quality code that can affect system reliability and performance (e.g., high resource usage, high execution time, etc.).


Figure \ref{fig:bbackground.pdf} summarizes the classical steps required to ensure the code generation and non-functional testing of produced code from design time to runtime. 
We distinguish four major steps: the software design using high-level system specifications, code generation by means of code generators, code execution, and non-functional testing of generated code. 


\begin{figure*}[t]
	\center
	
	\includegraphics[width=1.\linewidth]{chapitre4/fig/background.pdf}
	\caption{An overall overview of the different processes involved to ensure the code generation and non-functional testing of produced code from design time to runtime: the classical way}
	\label{fig:bbackground.pdf}
\end{figure*}


In the first step, software developers have to define, at design time, the software's behavior using a high-level abstract language (DSLs, models, program, etc). Afterwards, developers can use platform-specific code generators to ease the software development and automatically generate code for different languages and platforms. We depict, as an example in Figure \ref{fig:bbackground.pdf}, three code generators from the same family capable to generate code to three software programming languages (JAVA, C\# and C++). The first step is to generate code from the previously designed model.
% Transformations from model to code within each code generator might be different and may integrate different transformation rules. As an example, we distinguish model-to-model transformations languages such as ATL~\cite{jouault2005transforming} and template-based model-to-text transformation languages such as Acceleo~\cite{musset2006acceleo} to translate high-level system specifications into executable code and scripts~\cite{bragancca2008transformation,czarnecki2003classification}. The main task of code generators is to transform models to general-purpose and platform-dependent languages.
Afterwards, generated software artifacts (e.g., JAVA, C\#, C++, etc.) are compiled, deployed and executed across different target platforms (e.g., Android, ARM/Linux, JVM, x86/Linux, etc.). 
%Thus, several code compilers are needed to transform source code to machine code (binaries) in order to get executed. 
Finally, to perform the non-functional testing of generated code, developers have to collect, visualize and compare information about the performance and efficiency of running code across the different platforms. 
Therefore, they generally use several platform-specific profilers, trackers, instrumenting and monitoring tools in order to find some inconsistencies or bugs during code execution~\cite{guana2014chaintracker,delgado2004taxonomy}. Finding inconsistencies within code generators involves analyzing and inspecting the code and that, for each execution platform. For example, one way to handle that, is to analyze the memory footprint of software execution and find memory leaks~\cite{nethercote2007valgrind}. Developers can then inspect the generated code and find some fragments of the code-base that have triggered this issue. %Such non-functional error could occur when the code generator produces code that presents for example: incorrect typing, faulty memory management, code-smells, etc. 
Therefore, software testers generally use to report statistics about the performance of generated code in order to fix, refactor, and optimize the code generation process. Compared to this classical testing approach, our proposed work seeks to automate the last three steps: generate code, execute it on top of different platforms, and find code generator inconsistencies. 


\section{Approach overview}
\label{sec:cd_approach}
Now, we describe our approach overview. Our contributions in this work are divided in two parts:
\begin{itemize}
	\item First, we describe our testing infrastructure for the automatic code generation deployment and monitoring. This work is presented in more details in Chapter \ref{chap:docker}. This contribution addresses the problem of software diversity and hardware heterogeneity, as discussed in Chapter \ref{chap:background}.
	
	\item Second, we present a methodology for automatically detecting inconsistencies in a code generator family. This approach addresses the oracle problem when testing the resource usage and performance properties.
\end{itemize}


\subsection{An infrastructure for non-functional testing using system containers}
In this contribution, we focus on evaluating the non-functional properties related to the resource usage and performance of generated code. To do so, many system configurations (i.e., execution environments, libraries, compilers, etc.) must be taken into account to efficiently generate and test code. 

However, tunning different applications (i.e., generated code) with different configurations on one single machine is complex. A single system has limited resources and this can lead to performance regressions. Moreover, each execution environment comes with a collection of appropriate tools such as compilers, code generators, debuggers, profilers, etc. Therefore, we need to deploy the test harness, i.e., the produced binaries, on an elastic infrastructure that provide facilities to the code generator developers to ensure the deployment and monitoring of generated code in different environment settings. 
Consequently, the testing infrastructure should provide support to automatically:
\begin{enumerate}
	\item Deploy the generated code, its dependencies and its execution environments
	\item Execute the produced binaries in an isolated environment 
	\item Monitor the execution 
	\item Gather performance metrics (CPU, Memory, etc.)
\end{enumerate}

To ensure these four main steps, we rely on system containers~\cite{soltesz2007container} as a dynamic and customizable execution environment for running and evaluating the generated programs in terms of resource usage.

\begin{figure*}[h]
	\center
	\includegraphics[width=1.\linewidth]{chapitre5/fig/docker_background2.pdf}
	\caption{A technical overview of the different processes involved to ensure the code generation and non-functional testing of produced code from design time to runtime.}
	\label{fig:cg-infra}
\end{figure*}

Figure \ref{fig:cg-infra} shows the new container-based infrastructure used for testing code generators. Compared to the classical method presented in Figure \ref{fig:bbackground.pdf}, we added the following features: 
\begin{itemize}
	\item[--] At the code generation level: Code generators are configured inside different containers in order to generate code for the target platform.
	\item[--] At the code execution: Libraries, compilers and different dependencies are configured in different containers in order to execute the generated code. For each target platform a new instance is created.
	\item[--] At the non-functional level: We add a runtime monitoring engine (based on containers) in order to extract the resource usage properties.
\end{itemize}

Chapter \ref{chap:docker} provides more details about the technical choices we have made to synthesize this testing infrastructure.




\subsection{A metamorphic testing method for automatic detection of code generators inconsistencies}
Automatically detecting non-functional issues in code generators raises the oracle problem since there is no a clear definition of how the oracle might be defined and how we can determine the expected outcomes of selected test cases.

We discussed in section \ref{sec:cg-Summary: oracle definition approaches} of Chapter \ref{chap:SOTA} several approaches from the software testing community to alleviate the oracle problem. Among the attractive approaches that can be applied to test code generators, we distinguish the metamorphic testing approach (derived oracles). This approach is already applied for generators, especially for testing compilers\cite{donaldson2016metamorphic,tao2010automatic,le2014compiler}. In the following, we describe the basic concept of metamorphic testing and our adaptation of this testing approach to test the code generator families in terms of resource usage and performance.

\subsubsection{Basic concept of metamorphic testing}
In this section, we shall introduce the basic concept of metamorphic testing (MT). 
MT, proposed by Chen et al. \cite{chen1998metamorphic}, is a technique conceived to alleviate the oracle problem. It is based on the idea that often it is simpler to understand the relation between test cases' outputs rather than reasoning about the relation between test inputs and outputs. 

MT recommends that, given one or more test cases (called "source test cases", "original test cases", or "successful test cases") and their expected outcomes (obtained through multiple executions of the target program under test), one or more follow-up test cases can be constructed to verify the necessary properties (called "metamorphic relations" or "MRs") of the system or function to be implemented. In this case, the generation of the follow-up test cases and verification of the test results require the respect of the MR.

The classical example of MT is that of a program that computes the \textit{sin} function. A useful metamorphic relation for \textit{sin} functions is $\textit{sin(x) = sin($\pi$ - x)}$. Thus, even though the expected value for the source test case \textit{sin(50)} for example in not known, a follow-up test case can be constructed to verify the MR defined earlier. In this case, the follow-up test case is $\textit{sin($\pi$ - 50)}$ which must produce an output value that is equal to the one produced by the original test case \textit{sin(50)}. If this property is violated, then a failure is immediately detected.
MT generates follow-up test cases as long as the metamorphic relations are respected.
This is an example of a metamorphic relation: an input transformation that can be used to generate new test cases from existing test data, and an output relation (MR), that compares the outputs produced by a pair of test cases.

They can be any relations involving the inputs and outputs of two or more executions of the target program such as equalities, inequalities, periodicity properties, convergence constraints, subsumption relationships and many others. 

Because MT checks the relations among several executions rather than the correctness of individual outputs, MT can be used to fully automate the testing process without any manual intervention. 
However, constructing metamorphic relations is typically a manual task that demands thorough knowledge of the program under test. It also depends on the application context and domain. 
The effectiveness of metamorphic testing is highly dependent on the specific metamorphic relations that are used, and designing effective metamorphic relations is thus a critical step when applying metamorphic testing.

We describe in the next section our adaptation of MT to the problem of non-functional testing of code generators families.

\subsubsection{Application of MT to the non-functional testing of code generator families}
In general, MT can be applied to any problem in which a necessary property involving multiple executions of the target function can be formulated. Some examples of successful applications are presented in \cite{zhou2004metamorphic}. They includes the testing of simulation programs; the testing of numerical programs such as those for solving partial differential equations; the testing of graphics-rendering programs; and testing compilers.

To apply MT, there are four basic steps to follow:

\begin{enumerate}
 \item Find the properties of the system under test: the system should be investigated manually in order to find intended MRs defining the relation between inputs and outputs. This is based on the source test cases.
 \item Generate/select test inputs that satisfy the MR: this means that new the follow-up test cases must be generated or selected in order to verify their outputs using the MR.
 \item Execute the system with the inputs and get outputs: original and follow-up test cases are executed in order to gather their outputs.
 \item Check whether these outputs satisfy the MR, and if not, report failures.
\end{enumerate}

We develop now these four points in details in order show our MT adaptation to the code generators testing problem. 
\subsubsection[(Step 1)]{Metamorphic relation }

Step 1 consists in identifying necessary properties of the program under test and represent them as metamorphic relations among multiple test case inputs and their expected outputs.

In the context of code generators testing, we apply the concept of metamorphic testing described above to detect inconsistencies that violate MRs.
To do so, we have to define suitable MRs to automate this process.

As already stated, a metamorphic relation is a relation between different executions.
If we use the MR definition as presented in \cite{tao2010automatic,chan2006integration}:

\begin{mydef}[\textbf{Metamorphic relation}]

Let $(x_{1}, x_{2},..., x_{k})$ be a series of inputs to a function $f$, where $k$ $\geqslant$ 1, and $(f(x_{1}, x_{2},..., x_{k})$ be the corresponding series of results. Suppose $(f(x_{i1}), f(x_{i2}),..., f(x_{im}))$ is a subseries, possibly an empty subseries, of $(f(x_{1})$, $f(x_{2})$,..., $f(x_{k}))$. Let $(x_{k+1}, x_{k+2},..., x_{n})$ be another series of inputs to $f$, where $n \geqslant k+1$, and $(f(x_{k+1})$, $f(x_{k+2})$,..., $f(x_{xn}))$ be the corresponding series of results. Suppose, further, that there exists relations $r(x_{1}$, $x_{2}$,..., $x_{k}$, $f(x_{i1})$, $f(x_{i2})$,...., $f(x_{im})$, $x_{k+1}$, $x_{k+2}$,..., $x_{n}))$ and $r'$$(x_{1}$, $x_{2}$,..., $x_{n}$, $f(x_{1}$, $f(x_{2}$,..., $f(x_{n}))$ such that $r'$ must be true whenever $r$ is satisfied. We say that 


\textbf{MR} = {$(x_{1}, x_{2},...,x_{n}, f(x_{1}), f(x_{2}),..., f(x_{n}) \mid$

$r(x_{1}, x_{2},..., x_{k}, f(x_{i1}), f(x_{i2}),...., f(x_{im}), x_{k+1}, x_{k+2},..., x_{n}))$

$\Rightarrow r'(x_{1}, x_{2},..., x_{n}, f(x_{1}), f(x_{2}),..., f(x_{n}))$} 

is a metamorphic relation. When there is no ambiguity, we simply write the metamorphic relation as 	

\textbf{MR}: if $r(x_{1}, x_{2},..., x_{k}, f(x_{i1}), f(x_{i2}),...., f(x_{im}), x_{k+1}, x_{k+2},..., x_{n}))$

then $r'(x_{1}, x_{2},..., x_{n}, f(x_{1}), f(x_{2}),..., f(x_{n})).$

Furthermore, $x_{1}, x_{2},..., x_{k}$ are known as source test cases and $x_{k+1}, x_{k+2},..., x_{xn}$ are known as follow-up test cases
%Let (x1, x2,..., xk) be a series of inputs to a function f , where k > 1, and f(x1), f(x2),..., f(xk) be the corresponding series of results. Suppose f(xi1 ), f(xi2 ),..., f(xim ) is a subseries, possibly an empty subseries, of f(x1), f(x2),..., f(xk). Let xk+1, xk+2,..., xn be another series of inputs to f , where n ≥ k + 1, and f(xk+1), f(xk+2),..., f(xn) be the corresponding series of results. Suppose, further, that there exists relations r(x1, x2, ..., xk, f(xi1 ), f(xi2 ), ..., f(xim ), xk+1, xk+2, ..., xn) and r (x1, x2,..., xn, f(x1), f(x2),..., f(xn)) such that r must be true whenever r is satisfied. We say that MR = { (x1, x2,..., xn, f(x1), f(x2),..., f(xn)) | r(x1, x2,..., xk, f(xi1 ), f(xi2 ),..., f(xim ), xk+1, xk+2,..., xn) → r (x1, x2,..., xn, f(x1), f(x2),..., f(xn)) } is a metamorphic relation. When there is no ambiguity, we simply write the metamorphic relation as MR: If r(x1, x2,..., xk, f(xi1 ), f(xi2 ),..., f(xim ), xk+1, xk+2,..., xn) then r (x1, x2,..., xn, f(x1), f(x2),..., f(xn)). Furthermore, x1, x2, ..., xk are know

\end{mydef}

A code generator family can be looked as a function: $C : I \rightarrow P$, where $I$ is the domain of valid high-level source programs and $P$ is the domain of the target programs that are generated by the different code generators of the same family. The property of a code generator family implies that the generated programs $P$ share the same behavior as it is specified in $I$ using the high-level system specification. 


The availability of multiple generators with comparable functionality allow us to adapt MT in order to detect non-functional inconsistencies. In fact, if we can find out proper R relation of the non-functional behavior, we can get the metamorphic relation and conduct MT for testing code generator families.
Let $f(P(x))$ be a function that calculates a non-functional metric of a generated program $P$ (such as execution time or memory usage of $P$) for an input test suite denoted by $x$. Since we have different program versions generated in the same family, we denote by $(P_{1}(x)$, $P_{2}(x)$,..., $P_{n}(x))$ the set of generated programs. The corresponding outputs would be $(f(P_{1})$, $f(P_{2})$,..., $f(P_{n}))$. Thus, our MR looks like this:

\begin{equation}
	 R(P_{1}(x), P_{2}(x),..., P_{n}(x))  \Rightarrow R(f(P_{1}(x)), f(P_{2}(x)),..., f(P_{n}(x)))
\end{equation}

On the one hand, we use the following equation $P_{1}(x) \equiv P_{2}(x)$ to denote \textit{the functional equivalence relation} between two generated programs $P_{1}$ and $P_{2}$ from the same family. This means that the generated programs $P_{1}$ and $P_{2}$ have the same behavioral design, and for any test suite $x$, they have the same functional output. 
If this relation is not satisfied that means, that there is at least one faulty code generator that produced incorrect code. In this this work, we focus on the non-functional testing, so we ensure that this relation is ensured by excluding all the programs that do not exhibit the same behavior. 


On the other hand, since we are comparing equivalent implementations of the same program written in different languages, we assume that the memory usage and execution time should be more or less the same with a small variation for each test suite across the different versions. Obviously, we are expecting to get a variation between different executions because we are comparing the execution time and memory usage of test suites that are written in different languages and executed using different technologies (e.g., interpreters for PHP, JVM for JAVA, etc.). 
This observation is also based on initial experiments, where we evaluate the resource usage/execution time of several test suites across a set of equivalent versions generated using a code generator family (presented in details in the evaluation section \ref{sec:cg_evaluation}). As a consequence, we use the notation $\Delta\{f(P_{1}(x)), f(P_{2}(x))\}$ to designate the variation of memory usage or execution time of test suite execution $x$ across two version of generated code written in different languages $P_{1}$ and $P_{2}$. We suppose that this variation should note exceed a certain threshold value $T$, otherwise, we raise a code generator inconsistency.
Based on this intuition, the MR can be represented as:

\begin{equation}
P_{1}(x) \equiv P_{2}(x) \equiv ... \equiv P_{n}(x) \Rightarrow \Delta\{f(P_{1}(x)), f(P_{2}(x)),..., f(P_{n}(x))\} < T\quad (n\geqslant 2)
\end{equation}
%\paragraph{Generate test cases (Step 2)}
%\paragraph{Execute test cases (Step 3)}
%\paragraph{Detect inconsistencies (Step 4)}


\subsubsection[(Steps 2, 3, and 4)]{Metamorphic testing}
So far, we have defined the metamorphic property (MR) necessary for inconsistencies detection. We describe now our automatic metamorphic testing approach based on this relation (steps 2, 3, and 4). 
Figure 

Our approach is a black-box testing technique and it does not provide detailed information about the source of the issues, as described above. Nevertheless, we rather provide a mechanism to detect these potential issues within a set of code generator families so that, these issues may be investigated and fixed afterwards by code generators/software maintainers.


\begin{figure}[h]
	\centering
	\includegraphics[width=1.\linewidth]{chapitre4/fig/MT}
	\caption{The metamorphic testing approach for automatic detection of code generator inconsistencies}
	\label{fig:cg_MT}
\end{figure}




\section{Evaluation}
\label{sec:cg_evaluation}
So far, we have presented an automated approach for detecting inconsistencies within code generator families. In this section, we evaluate the implementation of our approach by explaining the design of our empirical study and the different methods we used to assess the effectiveness of our approach. 
The experimental material is available for replication purposes\footnote{\url{https://testingcodegenerators.wordpress.com/}}.
\subsection{Experimental setup}
\subsubsection{Code generators under test: Haxe compilers}
In order to test the applicability of our approach, we conduct experiments on a popular high-level programming language called Haxe and its code generators. Haxe is an open source toolkit for cross-platform development which compiles to a number of different programming platforms, including JavaScript, Flash, PHP, C++, C\# and Java. Haxe involves many features: the Haxe language, multi-platform compilers, and different native libraries. 
The Haxe language is a high-level programming language which is strictly typed. This language supports both functional programming and object-oriented programming paradigms. It has a common type hierarchy, making certain API available on every targeted platform.
Haxe comes with a set of compilers that translate manually-written code (in Haxe language) to different target languages and platforms. 
Haxe code can be compiled for applications running on desktop, mobile and web platforms. It comes also with a set of standard libraries that can be used on all supported targets and platform-specific libraries for each of them.

The process of code transformation and generation can be described as following: Haxe compilers analyze the source code written in Haxe language. Then, the code is checked and parsed into a typed structure, resulting in a typed abstract syntax tree (AST). This AST is optimized and transformed afterwards to produce source code for the target platform/language.
Haxe offers the option of choosing which platform to target for each program using command-line options. Moreover, some optimizations and debugging information can be enabled through command-line interface, but in our experiments, we did not turn on any further options. 

The Haxe code generators constitute the code generator family we would evaluate in this work.

\subsubsection{Cross-platform benchmark}
One way to prove the effectiveness of our approach is to create benchmarks. Thus, we use the Haxe language and its code generators to build a cross-platform benchmark. The proposed benchmark is composed of a collection of cross-platform libraries that can be compiled to different targets. In these experiments, we consider a code generator family composed of five target Haxe compilers: Java, JS, C++, CS, and PHP code generators. To select cross-platform libraries, we explore github and we use the Haxe library repository\footnote{\url{http://thx-lib.org/}}. So, we select seven libraries that provide a set of test suites with high code coverage scores. 

In fact, each Haxe library comes with an API and a set of test suites. These tests, written in Haxe, represent a set of unit tests that covers the different functions of the API. The main task of these tests is to check the correct functional behavior of generated programs. To prepare our benchmark, we remove all the tests that fail to compile to our five targets (i.e., errors, crashes and failures) and we keep only test suites that are functionally correct in order to focus only on the non-functional properties.
Moreover, we add manually new test cases to some libraries in order to extend the number of test suites. The number of test suites depends on the number of existing functions within the Haxe library.

We use then these test suites to transform functional tests into stress tests. This can be useful to study the impact of this load on the resource usage properties of the five target versions. For example, if one test suite consumes a lot of resources for a specific target, then this could be explained by the fact that the code generator under test has produced code that is very greedy in terms of resources.
Thus, we run each test suite 1K times to get comparable values in terms of resource usage.
Table \ref{tab:Description of selected benchmark libraries} describes the Haxe libraries that we have selected in this benchmark to evaluate our
approach and the number of test suites used per benchmark.

\begin{table}[h]
	\centering
	
		\begin{tabular}{|c|c|p{8.5cm}|}				
			\hline
			\textbf{Library} & \textbf{\#TestSuites} & \textbf{Description} \\
			\hhline{|=|=|=|}
			Color  &  19 &  Color conversion from/to any color space   \\ \hline
			Core & 51  & Provides extensions to many types  \\ \hline
			Hxmath & 6  & A 2D/3D math library  \\ \hline
			Format  &  4 & Format library such as dates, number formats   \\ \hline
			Promise & 5  & Library for lightweight promises and futures  \\ \hline
			Culture & 5  & Localization library for Haxe \\ \hline
			Math & 5  & Generation of random values \\ \hline
		\end{tabular}
	
	\caption{Description of selected benchmark libraries}
	\label{tab:Description of selected benchmark libraries}
\end{table}

\subsubsection{Evaluation metrics used}
We use to evaluate the efficiency of generated code using the following non-functional metrics:

-\textit{Memory usage}:
It corresponds to the maximum memory consumption of the running test suite. Memory usage is measured in \SI{}{\mega\byte}

-\textit{Execution time}:
Program execution time is measured in seconds.

We recall that our testing infrastructure is able to evaluate other non-functional properties of generated code such as code generation time, compilation time, code size, CPU usage. We choose to focus, in this experiment, on the performance (i.e., execution time) and resource usage (i.e., memory usage).

\subsubsection{Setting up infrastructure}

To assess our approach, we configure our previously proposed container-based infrastructure in order to run experiments on the Haxe case study.
Figure \ref{fig:settingup.pdf} shows a big picture of the testing and monitoring infrastructure considered in these experiments.

\begin{figure}[h]
	\centering
	\includegraphics[width=0.8\linewidth]{chapitre4/fig/settingup.pdf}
	\caption{Infrastructure settings for running experiments}
	\label{fig:settingup.pdf}
\end{figure}

First, a first component is created in where we install the Haxe code generators and compilers. It takes as an input the Haxe library we would test and the list of test suites (step 1). It produces as an output the source code for specific software platforms. These files are saved in a shared repository call \textit{Data Volume}.

Afterwards, generated files are compiled (if needed) and automatically executed within the execution container (Step 2). This execution container is a pre-configured container instance where we install all the required execution environments such as php interpreter, NodeJS, etc. 

In the meantime, while running test suites inside the container, we collect runtime resource usage data using the components showed in step 3, 4, and 5 (presented in details in Chapter \ref{chap:docker}).

Finally, in step 6 we provide a mechanism to extract these resource usage metrics via http requests. In our experiment, we are gathering the maximum memory usage values without presenting the graphs of resource usage profiles.

To obtain comparable and reproducible results, we use the same hardware across all experiments: a farm of AMD A10-7700K APU Radeon(TM) R7 Graphics processor with 4 CPU cores (\SI{2.0}{\GHz}), running Linux with a 64 bit kernel and \SI{16}{\giga\byte} of system memory. We reserve one core and \SI{4}{\giga\byte} of memory for each running container. 

\subsection{Experimental results}
\subsubsection{Evaluation using the standard deviation}
We now conduct experiments based on the new created benchmark. 
%We run each test suite 1K times and we report the execution time and memory usage across the different target languages: Java, JS, C++, CS, and PHP. 
The goal of running these experiments is to observe and compare the behavior of generated code in order to detect code generator inconsistencies.
%We recall, as mentioned in the motivation, that we are not using any oracle function to detect inconsistencies. However, we rely on the comparison results across different targets to detect code generator inconsistencies.

 
 
Therefore, we use, as a quality metric, the standard deviation to quantify the amount of variation among execution traces (i.e., memory usage or execution time) and that, for the five target languages. We recall that the formula of standard deviation is the square root of the variance. Thus, we are calculating this variance as the squared differences from the mean. Our data values in our experiment represent the obtained values in five languages. So, for each test suite we are taking the mean of these five values in order to calculate the variance.
A low standard deviation of a test suite execution, indicates that the data points (execution time or memory usage data) tend to be close to the mean which we consider as an acceptable behavior.  
On the other hand, a high standard deviation indicates that one or more data points are spread out over a wider range of values which can be more likely interpreted as a code generator inconsistency. 

In Table \ref{tab:The comparison results1}, we report the comparison results of running the benchmark in terms of execution speed. At the first glance, we can clearly see that all standard deviations are mostly close to 0 - 8 interval. However, we remark in the same table, that there are some variation points where the deviation is relatively high. We count 8 test suites where the deviation is higher than 60 (highlighted in gray). We choose this value (i.e., standard deviation = 60) as a threshold to designate the points where the variation is extremely high. Thus, we consider values higher than 60 as a potential possibility where a non-functional bug could occur. These variations can be explained by the fact that the execution speed of one or more test suites varies considerably from one language to another. This argues the idea that the code generator has produced a suspect code behavior for one or more target language, which led to a high performance variation. We provide later better explanation in order to detect the faulty code generators.

Similarly, Table \ref{tab:The comparison results2} resumes the comparison results of test suites execution regarding memory usage. The variation in this experiment are more important than previous results. This can be argued by the fact that the memory utilization and allocation patterns are different for each language. Nevertheless, we can recognize some points where the variation is extremely high. Thus, we choose a threshold value equal to 400 and we highlighted, in gray, the points that exceed this value. Thus, we detect 6 test suites where the variation is extremely high. 
One of the reasons that caused this variation may occur when the test suite executes some parts of the code (in a specific language) that are so greedy in terms of resources. This may not be the case when the variation is lower than 10 for example.
We assume then, that faulty code generators, in identified points, represent a threat for software quality since the generated code has shown symptoms of poor-quality design.

%The inconsistencies we are trying to find here are more related to the incorrect memory utilization patterns produced by the faulty code generator. Such inconsistencies may come from an inadequate type usage, high resource instantiation, etc.

\begin{table}[h]
	\centering
	
	\resizebox{\columnwidth}{!}{%
		\begin{tabular}{|l|l|S[table-format=3.2]|l|S[table-format=3.2]|l|S[table-format=3.2]|}
			\hline
			\textbf{Benchmark}                 & \textbf{TestSuite} & \textbf{Std\_dev}                & \textbf{TestSuite} & \textbf{Std\_dev}             & \textbf{TestSuite} & \textbf{Std\_dev}              \\ \hline
			& \textbf{TS1}       & 0.55                             & \textbf{TS8}       & 0.24                          & \textbf{TS15}      & 0.73                           \\ \cline{2-7} 
			& \textbf{TS2}       & 0.29                             & \textbf{TS9}       & 0.22                          & \textbf{TS16}      & 0.12                           \\ \cline{2-7} 
			& \textbf{TS3}       & 0.34                             & \textbf{TS10}      & 0.10                          & \textbf{TS17}      & 0.31                           \\ \cline{2-7} 
			& \textbf{TS4}       & 2.51                             & \textbf{TS11}      & 0.17                          & \textbf{TS18}      & 0.34                           \\ \cline{2-7} 
			& \textbf{TS5}       & 1.53                             & \textbf{TS12}      & 0.28                          & \textbf{TS19}      & \cellcolor[HTML]{C0C0C0}120.61 \\ \cline{2-7} 
			& \textbf{TS6}       & 43.50                            & \textbf{TS13}      & 0.33                         & \multicolumn{2}{l|}{\multirow{2}{*}{}} \\ \cline{2-5}
			\multirow{-7}{*}{\textbf{Color}}   & \textbf{TS7}       & 0.50                             & \textbf{TS14}      & 1.88                          & \multicolumn{2}{l|}{}                  \\ \hline
			& \textbf{TS1}       & 0.35                             & \textbf{TS18}      & 0.16                          & \textbf{TS35}      & 1.30                           \\ \cline{2-7} 
			& \textbf{TS2}       & 0.07                             & \textbf{TS19}      & 0.60                          & \textbf{TS36}      & 1.13                           \\ \cline{2-7} 
			& \textbf{TS3}       & 0.30                             & \textbf{TS20}      & 5.79                          & \textbf{TS37}      & 2.02                           \\ \cline{2-7} 
			& \textbf{TS4}       & \cellcolor[HTML]{C0C0C0}27299.89 & \textbf{TS21}      & 0.47                          & \textbf{TS38}      & 0.26                           \\ \cline{2-7} 
			& \textbf{TS5}       & 6.12                             & \textbf{TS22}      & 2.74                          & \textbf{TS39}      & 0.16                           \\ \cline{2-7} 
			& \textbf{TS6}       & 21.90                            & \textbf{TS23}      & 2.14                          & \textbf{TS40}      & 8.12                           \\ \cline{2-7} 
			& \textbf{TS7}       & 0.41                             & \textbf{TS24}      & 3.79                          & \textbf{TS41}      & 5.45                           \\ \cline{2-7} 
			& \textbf{TS8}       & 0.28                             & \textbf{TS25}      & 0.19                          & \textbf{TS42}      & 0.11                           \\ \cline{2-7} 
			& \textbf{TS9}       & 0.78                             & \textbf{TS26}      & 0.13                          & \textbf{TS43}      & 1.41                           \\ \cline{2-7} 
			& \textbf{TS10}      & 1.82                             & \textbf{TS27}      & 5.59                          & \textbf{TS44}      & 1.56                           \\ \cline{2-7} 
			& \textbf{TS11}      & \cellcolor[HTML]{C0C0C0}180.68   & \textbf{TS28}      & 1.71                          & \textbf{TS45}      & 0.11                           \\ \cline{2-7} 
			& \textbf{TS12}      & \cellcolor[HTML]{C0C0C0}185.02   & \textbf{TS29}      & 0.26                          & \textbf{TS46}      & 1.04                           \\ \cline{2-7} 
			& \textbf{TS13}      & \cellcolor[HTML]{C0C0C0}128.78   & \textbf{TS30}      & 0.44                          & \textbf{TS47}      & 0.23                           \\ \cline{2-7} 
			& \textbf{TS14}      & 0.71                             & \textbf{TS31}      & 1.71                          & \textbf{TS48}      & 1.34                           \\ \cline{2-7} 
			& \textbf{TS15}      & 0.12                             & \textbf{TS32}      & 2.42                          & \textbf{TS49}      & 1.86                           \\ \cline{2-7} 
			& \textbf{TS16}      & 0.65                             & \textbf{TS33}      & 8.29                          & \textbf{TS50}      & 1.28                           \\ \cline{2-7} 
			\multirow{-17}{*}{\textbf{Core}}   & \textbf{TS17}      & 0.26                             & \textbf{TS34}      & 5.25                          & \textbf{TS51}      & 3.53                           \\ \hline
			& \textbf{TS1}       & 31.65                            & \textbf{TS3}       & 30.34                         & \textbf{TS5}       & 0.40                           \\ \cline{2-7} 
			\multirow{-2}{*}{\textbf{Hxmath}}  & \textbf{TS2}       & 4.27                             & \textbf{TS4}       & 0.25                          & \textbf{TS6}       & 0.87                           \\ \hline
			& \textbf{TS1}       & 0.28                             & \textbf{TS3}       & \cellcolor[HTML]{C0C0C0}95.36 & \textbf{TS4}       & 1.49                           \\ \cline{2-7} 
			\multirow{-2}{*}{\textbf{Format}}  & \textbf{TS2}       & \cellcolor[HTML]{C0C0C0}64.94    & \multicolumn{4}{l|}{\textbf{}}                                                                           \\ \hline
			\textbf{Promise}                   & \textbf{TS1}       & 0.29                             & \textbf{TS2}       & 13.21                         & \textbf{TS3}       & 1.21                           \\ \hline
			& \textbf{TS1}       & 0.13                             & \textbf{TS3}       & 0.13                          & \textbf{TS4}       & 1.40                           \\ \cline{2-7} 
			\multirow{-2}{*}{\textbf{Culture}} & \textbf{TS2}       & 0.10                             & \multicolumn{4}{l|}{}                                                                                    \\ \hline
			\textbf{Math}                      & \textbf{TS1}       & \cellcolor[HTML]{C0C0C0}642.85   & \textbf{TS2}       & 28.32                         & \textbf{TS3}       & 24.40                          \\ \hline
		\end{tabular}%
	}
	
	\caption{The comparison results of running each test suite across five target languages: the metric used is the standard deviation between execution times  }
	\label{tab:The comparison results1}
\end{table}

% Please add the following required packages to your document preamble:
% \usepackage{multirow}
% Please add the following required packages to your document preamble:
% \usepackage{multirow}
% \usepackage[table,xcdraw]{xcolor}
% If you use beamer only pass "xcolor=table" option, i.e. \documentclass[xcolor=table]{beamer}
\begin{table}[]
	\centering
	
	\resizebox{\columnwidth}{!}{%
		\begin{tabular}{|l|l|S[table-format=3.2]|l|S[table-format=3.2]|l|S[table-format=3.2]|}
			\hline
			\textbf{Benchmark}                 & \textbf{TestSuite} & \textbf{Std\_dev}               & \textbf{TestSuite} & \textbf{Std\_dev}              & \textbf{TestSuite} & \textbf{Std\_dev}              \\ \hline
			& \textbf{TS1}       & 10.19                           & \textbf{TS8}       & 1.23                           & \textbf{TS15}      & 14.44                          \\ \cline{2-7} 
			& \textbf{TS2}       & 1.17                            & \textbf{TS9}       & 1.95                           & \textbf{TS16}      & 1.13                           \\ \cline{2-7} 
			& \textbf{TS3}       & 0.89                            & \textbf{TS10}      & 1.27                           & \textbf{TS17}      & 0.72                           \\ \cline{2-7} 
			& \textbf{TS4}       & 30.34                           & \textbf{TS11}      & 0.57                           & \textbf{TS18}      & 0.97                           \\ \cline{2-7} 
			& \textbf{TS5}       & 31.79                           & \textbf{TS12}      & 1.11                           & \textbf{TS19}      & \cellcolor[HTML]{C0C0C0}777.32 \\ \cline{2-7} 
			& \textbf{TS6}       & \cellcolor[HTML]{C0C0C0}593.05  & \textbf{TS13}      & 0.46                           & \multicolumn{2}{l|}{}                               \\ \cline{2-5}
			\multirow{-7}{*}{\textbf{Color}}   & \textbf{TS7}       & 12.14                           & \textbf{TS14}      & 45.90                          & \multicolumn{2}{l|}{\multirow{-2}{*}{}}             \\ \hline
			& \textbf{TS1}       & 1.40                            & \textbf{TS18}      & 1.00                           & \textbf{TS35}      & 14.13                          \\ \cline{2-7} 
			& \textbf{TS2}       & 1.17                            & \textbf{TS19}      & 20.37                          & \textbf{TS36}      & 32.41                          \\ \cline{2-7} 
			& \textbf{TS3}       & 0.60                            & \textbf{TS20}      & 128.23                         & \textbf{TS37}      & 22.72                          \\ \cline{2-7} 
			& \textbf{TS4}       & \cellcolor[HTML]{C0C0C0}403.15  & \textbf{TS21}      & 24.38                          & \textbf{TS38}      & 2.19                           \\ \cline{2-7} 
			& \textbf{TS5}       & 41.95                           & \textbf{TS22}      & 76.24                          & \textbf{TS39}      & 0.26                           \\ \cline{2-7} 
			& \textbf{TS6}       & 203.55                          & \textbf{TS23}      & 18.82                          & \textbf{TS40}      & 126.29                         \\ \cline{2-7} 
			& \textbf{TS7}       & 19.69                           & \textbf{TS24}      & 72.01                          & \textbf{TS41}      & 31.01                          \\ \cline{2-7} 
			& \textbf{TS8}       & 0.78                            & \textbf{TS25}      & 0.21                           & \textbf{TS42}      & 0.93                           \\ \cline{2-7} 
			& \textbf{TS9}       & 30.41                           & \textbf{TS26}      & 2.30                           & \textbf{TS43}      & 50.36                          \\ \cline{2-7} 
			& \textbf{TS10}      & 57.19                           & \textbf{TS27}      & 101.53                         & \textbf{TS44}      & 12.56                          \\ \cline{2-7} 
			& \textbf{TS11}      & 68.92                           & \textbf{TS28}      & 43.67                          & \textbf{TS45}      & 0.91                           \\ \cline{2-7} 
			& \textbf{TS12}      & 74.19                           & TS29               & 0.90                           & \textbf{TS46}      & 27.28                          \\ \cline{2-7} 
			& \textbf{TS13}      & 263.99                          & \textbf{TS30}      & 4.02                           & \textbf{TS47}      & 1.10                           \\ \cline{2-7} 
			& \textbf{TS14}      & 19.89                           & \textbf{TS31}      & 52.35                          & \textbf{TS48}      & 15.40                          \\ \cline{2-7} 
			& \textbf{TS15}      & 0.30                            & \textbf{TS32}      & 134.75                         & \textbf{TS49}      & 37.01                          \\ \cline{2-7} 
			& \textbf{TS16}      & 28.29                           & \textbf{TS33}      & 82.66                          & \textbf{TS50}      & 23.29                          \\ \cline{2-7} 
			\multirow{-17}{*}{\textbf{Core}}            & \textbf{TS17}      & 1.16                            & \textbf{TS34}      & 89.57                          & \textbf{TS51}      & 1.28                           \\ \hline
			& \textbf{TS1}       & \cellcolor[HTML]{C0C0C0}444.18  & \textbf{TS3}       & \cellcolor[HTML]{C0C0C0}425.65 & \textbf{TS5}       & 17.69                          \\ \cline{2-7} 
			\multirow{-2}{*}{\textbf{Hxmath}}  & \textbf{TS2}       & 154.80                          & \textbf{TS4}       & 0.96                           & \textbf{TS6}       & 46.13                          \\ \hline
			& \textbf{TS1}       & 0.74                            & \textbf{TS3}       & 255.36                         & \textbf{TS4}       & 8.40                           \\ \cline{2-7} 
			\multirow{-2}{*}{\textbf{Format}}  & \textbf{TS2}       & 106.87                          & \multicolumn{4}{l|}{\textbf{}}                                                                            \\ \hline
			\textbf{Promise}                   & \textbf{TS1}       & 0.30                            & \textbf{TS2}       & 58.76                          & \textbf{TS3}       & 20.04                          \\ \hline
			& \textbf{TS1}       & 1.28                            & \textbf{TS3}       & 0.58                           & \textbf{TS4}       & 15.69                          \\ \cline{2-7} 
			\multirow{-2}{*}{\textbf{Culture}} & \textbf{TS2}       & 4.51                            & \multicolumn{4}{l|}{}                                                                                     \\ \hline
			\textbf{Math}                      & \textbf{TS1}       & \cellcolor[HTML]{C0C0C0}1041.53 & \textbf{TS2}       & 234.93                         & \textbf{TS3}       & 281.12                         \\ \hline
		\end{tabular}%
	}
	
	\caption{The comparison results of running each test suite across five target languages: the metric used is the standard deviation between memory consumptions}
	\label{tab:The comparison results2}
\end{table}



\begin{table*}[h]
	\centering
	
	\resizebox{0.75\linewidth}{!}{%
		\begin{tabular}{|l|S[table-format=3.2]|S[table-format=3.2]|S[table-format=3.2]|S[table-format=3.2]|S[table-format=3.2]|S[table-format=3.2]|S[table-format=3.2]|S[table-format=3.2]|S[table-format=5.2]|S[table-format=3.2]|}
			\hline
			\multirow{2}{*}{}    & \multicolumn{2}{c|}{\textbf{JS}}   & \multicolumn{2}{c|}{\textbf{JAVA}} & \multicolumn{2}{c|}{\textbf{C++}}  & \multicolumn{2}{c|}{\textbf{CS}}   & \multicolumn{2}{c|}{\textbf{PHP}}  \\ \cline{2-11} 
			& \textbf{Time(s)} & \textbf{Factor} & \textbf{Time(s)} & \textbf{Factor} & \textbf{Time(s)} & \textbf{Factor} & \textbf{Time(s)} & \textbf{Factor} & \textbf{Time(s)} & \textbf{Factor} \\ \hline
			\textbf{Color\_TS19} & 4.52             & x1.0            & 8.61             & x1.9            & 10.73            & x2.4            & 14.99            & x3.3            & 279.27           & x61.8           \\ \hline
			\textbf{Core\_TS4}   & 665.78           & x1.0            & 416.85           & x0.6            & 699.11           & x1.1            & 1161.29          & x1.7            & 61777.21         & x92.8           \\ \hline
			\textbf{Core\_TS11}  & 4.27             & x1.0            & 1.80             & x0.4            & 1.57             & x0.4            & 5.71             & x1.3            & 407.33           & x95.4           \\ \hline
			\textbf{Core\_TS12}  & 4.71             & x1.0            & 2.06             & x0.4            & 1.60             & x0.3            & 5.36             & x1.1            & 417.14           & x88.6           \\ \hline
			\textbf{Core\_TS13}  & 6.26             & x1.0            & 5.91             & x0.9            & 11.04            & x1.8            & 14.14            & x2.3            & 297.21           & x47.5           \\ \hline
			\textbf{Format\_TS2}   & 2.31             & x1.0            & 2.10             & x0.9            & 1.81             & x0.8            & 6.08             & x2.6            & 148.24           & x64.1           \\ \hline
			\textbf{Format\_TS3}   & 5.40             & x1.0            & 5.03             & x0.9            & 7.67             & x1.4            & 12.38            & x2.3            & 220.76           & x40.9           \\ \hline
			\textbf{Math\_TS1}   & 3.01             & x1.0            & 12.51            & x4.2            & 16.30            & x5.4            & 14.14            & x4.7            & 1448.90          & x481.7          \\ \hline
		\end{tabular}%
	}
	\caption{Raw data values of test suites that led to the highest variation in terms of execution time}
	\label{tab:raw results1}
\end{table*}


\begin{table*}[h]
	\centering
	
	\resizebox{0.75\linewidth}{!}{%
		\begin{tabular}{|l|S[table-format=3.2]|S[table-format=3.2]|S[table-format=3.2]|S[table-format=3.2]|S[table-format=3.2]|S[table-format=3.2]|S[table-format=3.2]|S[table-format=3.2]|S[table-format=3.2]|S[table-format=3.2]|}
			\hline
			\multirow{2}{*}{}    & \multicolumn{2}{c|}{\textbf{JS}}      & \multicolumn{2}{c|}{\textbf{JAVA}}    & \multicolumn{2}{c|}{\textbf{C++}}     & \multicolumn{2}{c|}{\textbf{CS}}      & \multicolumn{2}{c|}{\textbf{PHP}}     \\ \cline{2-11} 
			& \textbf{Memory(Mb)} & \textbf{Factor} & \textbf{Memory(Mb)} & \textbf{Factor} & \textbf{Memory(Mb)} & \textbf{Factor} & \textbf{Memory(Mb)} & \textbf{Factor} & \textbf{Memory(Mb)} & \textbf{Factor} \\ \hline
			\textbf{Color\_TS6}  & 900.70              & x1.0            & 1362.55              & x1.5            & 2275.49             & x2.5            & 1283.31             & x1.4            & 758.79              & x0.8            \\ \hline
			\textbf{Color\_TS19} & 253.01              & x1.0            & 819.92              & x3.2            & 923.99              & x3.7            & 327.61              & x1.3            & 2189.86             & x8.7            \\ \hline
			\textbf{Core\_TS4}   & 303.09              & x1.0            & 768.22              & x2.5            & 618.42              & x2              & 235.75              & x0.8            & 1237.15             & x4.1            \\ \hline
			\textbf{Hxmath\_TS1} & 104.00              & x1.0            & 335.50              & x3.2            & 296.43              & x2.9            & 156.41              & x1.5            & 1192.98             & x11.5           \\ \hline
			\textbf{Hxmath\_TS3} & 111.68              & x1.0            & 389.73              & x3.5            & 273.12              & x2.4            & 136.49              & x1.2            & 1146.05             & x10.3           \\ \hline
			\textbf{Math\_TS1}   & 493.66              & x1.0            & 831.44              & x1.7            & 1492.97             & x3              & 806.33              & x1.6            & 3088.15             & x6.3            \\ \hline
		\end{tabular}%
	}
	\caption{Raw data values of test suites that led to the highest variation in terms of memory usage}
	\label{tab:raw results2}
\end{table*}



\subsubsection{Analysis}
Now that we have observed the non-functional behavior of test suites execution in different languages, we can analyze the extreme points we have detected in previous tables to observe in greater depth the source of such deviation.
For that reason, we present in Table \ref{tab:raw results1} and \ref{tab:raw results2} the raw data values of these extreme test suites in terms of execution time and memory usage. 

Table \ref{tab:raw results1} shows the execution time of each test suite in a specific target language. We also provide factors of execution times among test suites running in different languages by taking as a baseline the JS version. 
We can clearly see that the PHP code has a singular behavior regarding the performance with a factor ranging from x40.9 for test suite 3 in benchmark Format (Format\_TS3) to x481.7 for Math\_TS1. We remark also that running Core\_TS4 takes 61777 seconds (almost 17 hours) compared to a 416 seconds (around 6 minutes) in JAVA which is a very large gap. The highest factor detected for other languages ranges from x0.3 to x5.4 which is not negligible but it represents a small deviation compared to PHP version. While it is true that we are comparing different versions of generated code, it was expected to get some variations while running test cases in terms of execution time. However, in the case of PHP code generator, it is far to be a simple variation but it is more likely to be a code generator inconsistency that led to such performance regression.


Meanwhile, we gathered information about the points that led to the highest standard deviation in terms of memory usage. Table \ref{tab:raw results2} shows these results. %We take as well the JS version as a baseline since it requires less memory. 
Again, we can identify a singular behavior of the PHP code regarding the memory usage for the five last test suites with a factor ranging from x4.1 to x11.5 compared to the JS version.  For other test suites versions, the factor varies from x0.8 to x3.7. However, for Color\_TS6, C\# version consumes higher memory than other languages and even higher than PHP (x2.5 more than JS). %Besides the performance issues of PHP code generator presented in table~5, the results of memory usage confirm our claim since the PHP code has the highest memory utilization.
These results prove that the PHP code generator is not always effective and it constitutes a performance threat for the generated code.
%Our testing infrastructure allows to automatically detect inconsistencies among a set of code generators. 
This inconsistency need to be fixed later by code generator creators in order to enhance the code quality of generated code (PHP code for example). Since we are proposing a black-box testing approach, our solution is not able to provide more precise and detailed information about the part of code that has caused these performance issue, which is one of the limitations of this testing infrastructure.
Thus, to understand this particular singular performance of the PHP code when applying the test suite core\_TS4 for example, we looked (manually) into the PHP code corresponding to this test suite. In fact, we observe the intensive use of \textit{"arrays"} in most of the functions under test. Arrays are known to be slow in PHP and PHP library has introduced much more advanced functions such as $array\_fill$ and specialized abstract types such as \textit{"SplFixedArray"}\footnote{\url{http://php.net/manual/fr/class.splfixedarray.php}} to overcome this limitation. So, by changing just these two parts in the generated code, we improve the PHP code speed with a factor x5 which is very valuable. 


In short, the lack of use of specific types, in native PHP standard library, by the PHP code generator such as \textit{SplFixedArray} shows a real impact on the non-functional behavior of generated code. In contrast, selecting carefully the adequate types and functions to generate code can lead to performance improvement. We can observe the same kind of error in the C++ program  during one test suite execution (Color\_TS6) which consumes too much memory. The types used in the code generator are not the best ones. 

\subsection{Threats to validity}
We resume, in the following paragraphs, external and internal threats that can be raised:

\textit{External validity} refers to the generalizability of our findings. In this study, we perform experiments on Haxe and on a set of test suite selected from Github and from the Haxe community. For instance, we have no guarantee that these libraries cover all the Haxe language features neither than all the Haxe standard libraries. Consequently, we cannot guarantee that our approach is able to find all the code generators issues unless we develop a more comprehensive test suite. Moreover, the threshold defined to detect the singular performance behavior has a huge impact on the precision and recall of the proposed approach. Experiments should be replicated to other case studies to confirm our findings and try to understand the best heuristic to detect the code generator issues regarding performance (i.e., automatically calculate the threshold values)

\textit{Internal validity} is concerned with the use of a container-based approach. Even if it exists emulators such as Qemu\footnote{\url{https://goo.gl/SxKG1e}} that allow to reflect the behavior of heterogeneous hardware, the chosen infrastructure has not been evaluated to test generated code that target heterogeneous hardware machines. In addition, even though system containers are known to be lightweight and less resource-intensive compared to full-stack virtualization, we would validate the reliability of our approach by comparing it with a non-virtualized approach in order to see the impact of using containers on the accuracy of the results.



%hrough these conducted experiments, we reached interesting results, some of which were unexpected.


%In most languages, arrays are fixed-sized, but this is not the case in PHP since they are allocated dynamically. The dynamic allocation of arrays  leads to a slower write time because the memory locations needed to hold the new data is not already allocated. Thus, slow writing speed damages the performance of PHP code and impact the memory usage. This observation clearly confirm our early findings. The solution for this problem may be to use another type of object from the Standard PHP Library. As an alternative, \textit{"SplFixedArray"} pre-allocates the necessary memory and allows a faster array implementation, thereby solving the issue of slower write times. [ref to spl benchs]

%\subsection{Discussions}
%For instance, the poor performance of the Haxe-produced PHP code was certainly a surprise. 
%The PHP target had a very large execution time and memory usage for almost every test suite execution. 
%That is to say that PHP code generator is clearly generating non-efficient code regarding the non-functional properties. It is even possible to say that other code generators are not extremely efficient since we found that C++ code consumed, during one test suite execution (Color\_TS6), more memory than PHP. But, we cannot say for sure that C++ code generator is buggy. Thus, we cannot make any assumption. Nevertheless, the only point which, at this stage, can be statistically made is that PHP code generator has a real performance issue that has to be fixed by code generators developers. 

%In attempting to understand the reasons of this PHP performance issues, we tried to take a look at the source code generated in PHP. As an example, we looked at the source code of Core\_TS4. 




\section{Conclusion}
\label{sec:cg-conclusion}
In this work we have described a new approach for testing and monitoring the code generators families using a container-based infrastructure. 
We used a set of micro-services in order to provide a fine-grained understanding of resource consumption. 
To validate our approach, we evaluate a popular family of code generators: HAXE. 
The evaluation results show that we can find real issues in existing code generators. 
In particular, we show that we could find two kinds of errors: the lack of use of a specific function and an abstract type that exist in the standard library of the target language which can reduce the memory usage/execution time of the resulting program.

As a current work, we are discussing with the Haxe community to submit a patch with the first findings. 
We are also conducting the same evaluation for two other code generators families: ThingML and TypeScript. 
As a future work, we are going to improve our understanding on the threshold which can provide a best precision for detecting performance issues in code generators. 
In this paper, we detected inconsistencies related to the execution speed and memory usage. In the future, we seek, using the same testing infrastructure, to detect more code generator inconsistencies related to other non-functional metrics such CPU consumption, etc. 






















