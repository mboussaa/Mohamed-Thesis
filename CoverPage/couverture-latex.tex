\documentclass[11pt, english, openright]{book}
\usepackage{etex}
%% En 12pt c'est possible aussi
%% packages utilises
%%---------------------
\usepackage[utf8]{inputenc}
\usepackage[T1]{fontenc}
\usepackage{amsmath}
\usepackage{listings}
%\usepackage{math}
\usepackage{amssymb}
\usepackage{these}
\usepackage{setspace}
\usepackage{emptypage}
 \usepackage{array}
\usepackage[inline]{enumitem}
\usepackage{stmaryrd} %% llbracket & rrbracket
\usepackage{tabularx}
\usepackage{floatflt,amssymb}
\usepackage{graphicx}
\usepackage{moreverb} %% pour le verbatim en boite
%\usepackage{slashbox} %% pour couper les colonnes des tableaux en diagonale
%\usepackage{showkeys} %% pour voir les labels
\usepackage{multirow} %% pour regrouper un texte sur plusieurs lignes dans une table
\usepackage{url} %% pour citer les url par \url
\usepackage[all]{xy} %% pour la barre au dessus des symboles
\usepackage{shorttoc} %% pour plusieurs tables des matières par la commande \shorttableofcontents{Titre}{profondeur}.
\usepackage{textcomp} %% pour le symbol pour mille par \textperthousand.
\usepackage{tikz}
%\usepackage[colorlinks=true, linkcolor=blue, citecolor=blue]{hyperref} %% pour la transformation en PDF, ça permet d'obtenir des liens sur les sections ...
\usepackage[colorlinks=true, citecolor=blue, linkcolor=black, urlcolor=black]{hyperref}
%\usepackage{textcomp}
\usepackage[right]{eurosym}
\usepackage{xspace}
\usepackage{enumitem}
%\usepackage{eurosans} %%pour le symbole \euro
%\usepackage{epic,eepic}
%\usepackage{times}%times,palatino}
%\usepackage{palatino}
%\usepackage{helvetica} : tjrs le pb
%\usepackage{utopia}:pb de cesure
\usepackage{longtable}
\usepackage{rotating}
\usepackage{pifont}
\usepackage{subfig}
%\usepackage{subfigure}
%% macro/racourcis por les symboles et commandes usuelles


%% choix des profondeurs de section pour la table des matières
%% 2= subsection, 3=subsubsection
\setcounter{secnumdepth}{3}  %% Avec un numero.
\setcounter{tocdepth}{2}     %% Visibles dans la table des matieres

\makeindex %% crée l'index
\def\underscore{\char`\_}
\newcommand\ok[0]{\small \ding{51}}
\newcommand{\ie}{\emph{i.e.}\xspace}
\newcommand{\eg}{\emph{e.g.}}
\newcommand{\cf}{\emph{cf.}}
\newcommand{\etal}{\emph{et al.}\xspace}


\newcommand{\sconfig}[1]{\llbracket #1 \rrbracket}
\newcommand\productName[1]{\small #1}
\newcommand\ftProduct[1]{\small #1}
\newcommand\yes[0]{\small \ding{51}}
\newcommand\no[0]{\small \ding{53}}
\newcommand{\feature}[1]{{\sf{{\small{\textsf{#1}}}}}}
\newcommand{\myexample}[1]{	\emph{\small Example: #1}}
\renewcommand{\myexample}[1]{\emph{Example.} #1}

\newcommand{\opencompare}{{\small{\textsf{OpenCompare}}}\xspace}
\newcommand*\circled[1]{\tikz[baseline=(char.base)]{
            \node[shape=circle,draw,inner sep=0.4pt] (char) {#1};}}
            
\usepackage{todonotes}
\newcommand\mytodo[1]{\todo[inline]{#1}}

\newcommand{\hiddensubsection}[1]{
    \stepcounter{subsection}
    \subsection*{\arabic{chapter}.\arabic{section}.\arabic{subsection}\hspace{1em}{#1}}
}
            
%% c'est parti mon kiki !!
%%%%%%%%%%%%%%%%%%%%%%%%%%%%%%%%%%%%%%%%%%%
\begin{document}%%%%%%%%%%%%%%%%%%%%%%%%%%%
%%%%%%%%%%%%%%%%%%%%%%%%%%%%%%%%%%%%%%%%%%%
\frontmatter
\addtocounter{page}{-1}%ça c'est pour revenir à 0
%\fontfamilly{phv}

%%  1ere de Couverture:

\hyphenation{Modeling} 
\hyphenation{Variability} 


\titre{
\begin{flushleft}
Automatic Non-functional Testing and Tuning of Configurable Generators
\end{flushleft}}


%\soutenue
%%   Laisser cette ligne en commentaire sauf pour la version finale.
%%   (la premiere page contiendra "a soutenir le ..."
%%   au lieu de "soutenue le ...")


%% Les différents champs de la couverture...
\datesout{6 Septembre 2017}
\Auteur{Mohamed}{BOUSSAA}

% \Labo{IRISA -- UMR6074}
\Labo{INRIA}
\LaboEtendu{INRIA Rennes Bretagne Atlantique \\} % sauf INSA
\ComposanteUniversitaire{ISTIC} % sauf INSA

%% La composition du jury : prénom, nom, titre
%\President[Mme]{Pascale}{S\'ebillot}{Professeur, INSA de Rennes}         %% le président du jury
 
\Examinateur{Erven}{Rohou}{Directeur de recherche, INRIA Rennes} 
\Rapporteur[Mme]{H\'{e}l\`{e}ne}{Waeselynck}{Directrice de recherche, LAAS-CNRS Toulouse}
\Rapporteur{Philippe}{Merle}{Chargé de recherche, INRIA Lille}
\Examinateur{Franck}{Fleurey}{Directeur de recherche, SINTEF Oslo}
\Examinateur{Jean-Marie}{Mottu}{Ma\^itre de conférences, Universit\'e de Nantes} 
\Examinateur{Gerson}{Suny\'e}{Ma\^itre de conférences, Universit\'e de Nantes} 
\Advisor{Benoit}{Baudry}{Directeur de recherche, INRIA Rennes}
\Advisor{Olivier}{Barais}{Professeur, Universit\'e de Rennes 1}


\makethese{rennes1}



\end{document}